\documentclass[12pt,a4paper,final]{report}			%definire grandezza testo, foglio e tipo di testo.
\usepackage[left=2.6cm,right=2.6cm]{geometry}
\usepackage{pdf14}
\usepackage[utf8]{inputenc}			%definisce la codifica dei caratteri
\usepackage[italian]{babel}				%definisce il pacchetto della lingua
\usepackage{amsmath}						%contiene molti utili strumenti per la scrittura matematica
\usepackage{amsfonts}
\usepackage{amssymb}
\usepackage{makeidx}
\usepackage{units} 					%unita di misura
\usepackage{subcaption}  		%per le figure
\usepackage{subcaption}		%DON'T KNOW. INVESTIGATE
\usepackage{siunitx} 				%unita SI
\usepackage{mathrsfs}			%per usare cose tipo \mathscr{}
\usepackage{physics}				%contiene molta notazione utile; lo uso in particolare per \bra e \ket
\usepackage{cite}
\usepackage{comment}
\usepackage{caption}
\usepackage[font=small,labelfont=bf]{caption}		% serve per ridurre la dimensione delle captions
\usepackage[final]{pdfpages}		%per poter includere pdf
\usepackage[a-1a]{pdfx}
\usepackage[pdfa]{hyperref}
%\pdfminorversion=4

\title{Discorso Presentazione}

\begin{document}
\begin{center}
	\LARGE \textbf{Presentazione di Tesi}
	%\textbf{Stima delle proprietà di sistemi ottici nelle microonde tramite l'utilizzo di reti neurali.}
\end{center}
\begin{center}
	\normalsize \textit{Eleonora Gatti}
\end{center}

\vspace{5mm}


\noindent \textbf{Slide 1}:\\
Oggi vi esporrò il mio lavoro di tesi il cui titolo è "Stima dei parametri..."

\vspace{3mm}
\noindent \textbf{Slide 2:}\\
Durante questa presentazione inizierò definendo cos'è il diagramma di radiazione e descrivendo i parametri ho utilizzato durante il lavoro di tesi, vi parlerò della simulazione di sistemi ottici e passerò poi a descrivere le tecniche che ho utilizzato per la stima dei parametri di interesse, ovvero l'interpolazione e le reti neurali. Infine vi mostrerò i principali risultati e concluderò proponendo alcuni sviluppi futuri.


\vspace{3mm}
\noindent \textbf{Slide 3:}\\
Un'antenna puntata in una certa direzione nel cielo riceve (o trasmette) radiazione anche da direzioni diverse dalla direzione di vista.

Il diagramma di radiazione fornisce la risposta angolare di un'antenna, ovvero la potenza ricevuta o trasmessa in in funzione di una posizione $(\theta,\,\phi)$ che definisce quindi una posizione nel cielo.

\vspace{3mm}
\noindent \textbf{Slide 4:}\\

\vspace{3mm}
\noindent \textbf{Slide 5:}\\

\vspace{3mm}
\noindent \textbf{Slide 6:}\\
Se per STRIP, che presenta poche antenne sulla superficie focale, è ancora possibile efffettuare una simulazione completa dell'ottica, per altri esperimenti come LiteBIRD questo non è più possibile. Strip: 49 antenne a 43GHz e 6 antenne a 90GHz. LiteBIRD $~10^3$

\vspace{3mm}
\noindent \textbf{Slide 7:}\\

\vspace{3mm}
\noindent \textbf{Slide 8:}\\


\end{document}
